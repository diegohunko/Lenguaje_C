\documentclass{article}
\usepackage{verbatim}
\begin{document}
1.a)El proceso hijo imprime Sistemas y el proceso padre imprime Operativos:
\begin{verbatim}
Sistemas
Operativos
\end{verbatim}

1.b)La salida es:
\begin{verbatim}
Sistemas
Operativos
Operativos
\end{verbatim}

1.c) se imprimen diez ``1'' y diez ``2''.

\begin{verbatim}
                      _________pid1=frok()_________
                     /                             \
                    /                               \
               pid2=fork()                     pid2=fork()
                /         \                      /         \
               1           1                     x         x
               x           2                     x         2
      pid1=fork()        pid1=fork()          pid1=fork()  pid1=fork()
        /      \           |       |            |      |     |       |
pid2=fork() pid2=fork() pid2=fork  pid2=fork() pid2   pid2  pid2    pid2
 |       |    |     |      |           |       |  |   |  |  |  |    |  |
 1       1    x     x      x           x       1  1   x  x  1  1    x  x
 x       2    x     2      x           2       x  2   x  2  x  2    x  2
\end{verbatim}

3.a)Al no fallar ``execlp ("kcalc", "kcalc", NULL);''
la imagendel proceso es reemplazada por ``kcalc''.\\

3.b)
\begin{verbatim}
#include<sys/types.h>
#include<sys/wait.h>
#include<unistd.h>
#include<stdio.h>
#include<stdlib.h>
int main (int argc, char *argv[]) {
        pid_t pid = fork();
        if(pid == 0){
            execlp ("kcalc", "kcalc", NULL);
            printf ("Ap. 1 ejecutada\n");
        }else{
            execlp ("xload", "xload", NULL);
            printf ("Ap. 2 ejecutada\n");
        }
        return 0;
}
\end{verbatim}

\end{document}
