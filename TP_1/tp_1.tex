\documentclass{article}
\usepackage{verbatim}
\usepackage{moreverb}
\usepackage[spanish]{babel}
\title{\Huge{Gesti\'on de Procesos en C}}
\author{\Large{Fern\'andez Hunko, Diego Ra\'ul}}

\begin{document}
\maketitle

\begin{tabular}{l l}
\textbf{\large{Materia}} & \large{Sistemas Operativos} \\
\textbf{\large{Profesor Titular}} & \large{Ing. Rub\'en L.M.Casta\~no} \\      
\textbf{\large{JTP}} & \large{Ing. Roberto A. Mi\~no} \\
\textbf{\large{Ayudante 1ra}} & \large{Lic. Claudio O. Biale} \\
\end{tabular}
\newpage

1.a)El proceso hijo imprime Sistemas y el proceso padre imprime Operativos:
\begin{verbatim}
Sistemas
Operativos
\end{verbatim}

1.b)La salida es:
\begin{verbatim}
Sistemas
Operativos
Operativos
\end{verbatim}

1.c) se imprimen diez ``1'' y diez ``2''.

\begin{verbatim}
                      _________pid1=frok()_________
                     /                             \
                    /                               \
               pid2=fork()                     pid2=fork()
                /         \                      /         \
               1           1                     x         x
               x           2                     x         2
      pid1=fork()        pid1=fork()          pid1=fork()  pid1=fork()
        /      \           |       |            |      |     |       |
pid2=fork() pid2=fork() pid2=fork  pid2=fork() pid2   pid2  pid2    pid2
 |       |    |     |      |           |       |  |   |  |  |  |    |  |
 1       1    x     x      x           x       1  1   x  x  1  1    x  x
 x       2    x     2      x           2       x  2   x  2  x  2    x  2
\end{verbatim}

2.a)
\begin{verbatim}
               fork()
               /    \
            fork()       fork()
            /   \        /    \
          fork  fork   fork  fork
         /  \    / \    / \   /  \
         f  f    f  f  f  f  f   f
        /\  /\  / \ /\ /\ /\ /\  /\
        p p p p p p p ppp pp pp  p p
\end{verbatim}

2.b)
\begin{listing}{1}
#include <sys/types.h>
#include <unistd.h>
#include <stdio.h>
int main(int argc, char * argv[]) {
    int i, cant;
    pid_t hijo;                            (*)
    cant = atoi(argv[1]);
    for (i = 0; i <= cant; i++) {
        if((hijo = fork()) == 0){          (*)
            printf("%d\n", i);
            exit(0);                       (*)
        }
    }
}
\end{listing}

2.c) Si el usuario ejecuta \emph{./cp1 4} se crean 5 procesos hijos.

2.d)
\begin{listing}{1}
#include <sys/types.h>
#include <unistd.h>
#include <stdio.h>
int main(int argc, char * argv[]) {
    int i, cant;
    cant = atoi(argv[1]);
    pid_t hijo, fin[cant];                 (*)
    
    for (i = 0; i <= cant; i++) {
        if((hijo = fork()) == 0){          (*)
            printf("%d\n", i);
            exit(0);                       (*)
        }
    }
    
    for (i = 0; i <= cant; i++){           (*)
        fin[i]=wait(&estado);              (*)
        printf("Termino hijo: %d", fin[i]);(*)
    }
}
\end{listing}

3.a)Al no fallar ``execlp ("kcalc", "kcalc", NULL);''
la imagen del proceso es reemplazada por \emph{``kcalc''}.\\

3.b)
\begin{listing}{1}
#include<sys/types.h>
#include<sys/wait.h>
#include<unistd.h>
#include<stdio.h>
#include<stdlib.h>
int main (int argc, char *argv[]) {
        pid_t pid = fork();
        if(pid == 0){
            execlp ("kcalc", "kcalc", NULL);
            printf ("Ap. 1 ejecutada\n");
        }else{
            execlp ("xload", "xload", NULL);
            printf ("Ap. 2 ejecutada\n");
        }
        return 0;
}
\end{listing}
3.b)
\begin{listing}{1}
#include<sys/types.h>
#include<sys/wait.h>
#include<unistd.h>
#include<stdio.h>
#include<stdlib.h>
int main (int argc, char *argv[]) {
        pid_t hijo1, hijo2, fin1, fin2;
        int estado1, estado2; (*)
        
        if((hijo1 = fork()) == 0){
            execlp ("kcalc", "kcalc", NULL);
            printf ("Ap. 1 ejecutada\n");
        }
        if ((hijo2 = fork()) == 0){
            execlp ("xload", "xload", NULL);
            printf ("Ap. 2 ejecutada\n");
        }
        fin1 = wait(&estado1); (*)
        fin2 = wait(&estado2); (*)
        printf("termino el proceso: %d, estado: %d\n", fin1, 
                                  WEXITSTATUS(estado1)); (*)
        printf("termino el proceso: %d, estado: %d\n", fin2,
                                  WEXITSTATUS(estado2)); (*)
        return 0;
}
\end{listing}

3.d
\begin{verbatim}
                              hijo1=fork()
                              /          \
                             /            \
                       hijo2=fork()      execlp(kcalc)
                       /         \
                 fin1=wait()    execlp(xload)
                 fin2=wait()       /     \
                 printf()        proc1   proc2
                 printf()
\end{verbatim}

4)``Parcial.'' se imprime 22 veces.
5) Si \emph{execlp} no falla, se imprime:\\
\begin{verbatim}
Inicio
Valor hola
Hijo terminado
\end{verbatim}
 Si \emph{execlp} falla, se imprime:\\
\begin{verbatim}
Inicio
Final
Hijo terminado
\end{verbatim}
6)Tanto \textbf{hijo1} como \textbf{hijo2} se convierten en hu\'erfanos\\
10) Los procesos zombies son: hijo1 e hijo2. hijo4 queda hu\'erfano.
\end{document}
